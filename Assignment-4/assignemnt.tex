\documentclass[10pt,twocolumn]{IEEEtran}
\usepackage{amsfonts}
\usepackage{amsmath}
\usepackage{amssymb}
\usepackage[latin1]{inputenc}                                 
\usepackage{color}                                            
\usepackage{array}                                            
\usepackage{longtable}                                        
\usepackage{calc}                                             
\usepackage{multirow}                                         
\usepackage{hhline}                                           
\usepackage{ifthen}
\usepackage{enumitem}
\usepackage{graphicx}
\def\inputGnumericTable{}
\newcommand{\question}{\noindent \textbf{Question: }}
\newcommand{\solution}{\noindent \textbf{Solution: }}


\title{Assignment 4}
\author{Suraj kumar \\ \normalsize AI21BTECH11029\\ \Large CBSE 12th Ex 13.1 Q10 }
\begin{document}
\maketitle
\question A black and a red ball is rolled.
\begin{enumerate}[label=(\roman*)]
    \item Find the conditional probability of obtaining a sum greater than 9,given that black die resulted in a 5
    \item Find the conditional probability of obtaining the sum 8,given that the red die resulted in a number less than 4
\end{enumerate}
\solution
\begin{enumerate}[label=(\roman*)]
    \item  let, B denote black coloured die and R denote red colored die
          then, the sample space for the given experiment will be:\\
          S$=$ \\
          \{ (B1,R1),(B1,R2),(B1,R3),(B1,R4),(B1,R5),(B1,R6), \\
          (B2,R1),(B2,R2),(B2,R3),(B2,R4),(B2,R5),(B2,R6), \\
          (B3,R1),(B3,R2),(B3,R3),(B3,R4),(B3,R5),(B3,R6), \\
          (B4,R1),(B4,R2),(B4,R3),(B4,R4),(B4,R5),(B4,R6), \\
          (B5,R1),(B5,R2),(B5,R3),(B5,R4),(B5,R5),(B5,R6), \\
          (B6,R1),(B6,R2),(B6,R3),(B6,R4),(B6,R5),(B6,R6)\}\\
          (a) let $A$ be the event of 'obtaining a sum greater than 9' and $B$ be the event of ' getting 5 on black die'
          then $A=$
          \{(B4,R6),(B5,R5),(B5,R6),(B6,R4) \\
          ,(B6,R5),(B6,R6)\} \\
          and B=
          \{(B5,R1),(B5,R2),(B5,R3),        \\
          (B5,R4),(B5,R5),(B5,R6)\}
          \begin{align*}
              \Rightarrow  A \cap B = \{ (B5,R5),(B5,R6)\}
          \end{align*}
          So,
          \begin{align}
              P(A)=\frac{6}{36}=\frac{1}{6},\\
               P(A\cap B)=\frac{2}{36}=\frac{1}{19}
          \end{align}
          Now we know that by defination of conditional probability,
          \begin{align*}
              P\left(\frac{A}{B}\right) =\frac{P(A\cap B)}{P(B)}
          \end{align*}
          Now substiting the value we get
          \begin{align}
              \Rightarrow P\left(\frac{A}{B}\right)=\frac{\frac{1}{18}}{\frac{1}{6}}=\frac{6}{18}=\frac{1}{3}
          \end{align}
    \item let, A be the event of
          obtaining a sum 8
          and B be the event of 'getting a number less than 4 on red die'\\
          then $ A=$\{(B2,R6),(B3,R5),(B4,R4),\\(B5,R3),(B6,R2) \}\\
          B= \{(B1,R1),(B2,R1),(B3,R1),(B4,R1),(B5,R1),(B6,R1),\\
          (B1,R2),(B2,R2),(B3,R2),(B4,R2),(B5,R2),(B6,R2),\\
          (B1,R3),(B2,R3),(B3,R3),(B4,R3),(B5,R3),(B6,R3)\}\\
          and,
          $\Rightarrow A\cap B = \{ (B5,R3),(B6,R2)\} $\\
          So,
          \begin{align}
              P(A)=\frac{5}{36}                          \\
              P(B)=\frac{18}{36}=\frac{1}{2},            \\
              P(A\cap B)=\frac{2}{36}=\frac{1}{19}
          \end{align}
          so,
          we know that by conditional probability,
          \begin{align*}
              P\left(\frac{A}{B}\right)=\frac{P(A\cap B)}{P(B)}
          \end{align*}
          Now by substiting the value we get
          \begin{align}
              \Rightarrow P\left(\frac{A}{B}\right)=\frac{\frac{1}{18}}{\frac{1}{2}}=\frac{2}{18}=\frac{1}{9}
          \end{align}

\end{enumerate}

\end{document}