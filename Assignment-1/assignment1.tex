\documentclass[journal,12pt,twocolumn]{IEEEtran}

\usepackage{tfrupee}
\usepackage{enumitem}
\usepackage{amsmath}
\usepackage{amssymb}
\newcommand{\myvec}[1]{\ensuremath{\begin{pmatrix}#1\end{pmatrix}}}



\title{Assignment 1}
\author{Suraj kumar \\ \normalsize AI21BTECH11029\\\vspace*{20pt} \Large ICSE 2019 Grade 10}

\setlength{\columnsep}{1cm}
\begin{document}
\maketitle
\textbf{Question 7(c)} $\Rightarrow$ Given $\myvec{
     4 & 2 \\
     -1 & 1 \\
      }M=6I $,where M is a matrix and I is unit matrix of order $2\times 2.$
		\begin{enumerate}[label=(\roman*)]
		\item State the order of matrix M
		\item Find the matrix M
	\end{enumerate}	
 \textbf{solution (i) } 
 \begin{equation}
  \Rightarrow  \hspace{3mm} \myvec{
     4 & 2 \\
     -1 & 1 
      }M=6I
      \end{equation}
    here I is unit matrix.\\
    we have to find order of matrix M\\
    let the order the matrix M is  $$ \Rightarrow \hspace{3mm} a\times b$$
      
      we know that for multiply two matrix their order must be in the form of  (x,y) (y,z)  here $x,y,z \in N$
      hence order of matrix will be $2\times b$.\\
      so overall left hand side order is  $$\Rightarrow \hspace{3mm} (2\times 2)\times(2\times b) =(2\times b)$$
      for comparing LHS=RHS their order must be same so \\
      order of LHS$=(2\times b)$,
      RHS=$(2\times 2)$  hence b = 2.\\
      hence the order of matrix M is $(2\times 2).$\\\\

\textbf{solution (ii)}\\
\begin{equation}
    A=\myvec{
      4 & 2\\
      -1 & 1
     }
     \end{equation}
     \begin{equation}
       AM=6I
     \end{equation}
    \begin{equation}
    \text{multiply by}\hspace{3mm} A^{-1}
    \end{equation}
    \begin{equation}
    M=A^{-1}\times 6I \hspace{5mm} \therefore I\times M=M
    \end{equation}
    as we know that  $M \times I=I\times M^{-1}$
    
    
    
   	\begin{equation}	
   		\myvec{4 &2 & \vrule &1&0\\ -1&1&\vrule&0&1}
	\end{equation} 
	\begin{equation}
	    		R_2\rightarrow 4R_2+R_1 \hspace{10mm}  \myvec{4 &2 & \vrule 					&1&0\\ 0&6&\vrule&1&4}
	\end{equation}
	\begin{equation}
		R_1\rightarrow 3R_1-R_2 \hspace{10mm}  \myvec{12 &0 & \vrule 					&2&-4\\ 0&6&\vrule&1&4}
\end{equation}
\begin{equation}
		R_1\rightarrow \frac{R_1}{2} \hspace{10mm}  \myvec{6 &0 & \vrule 					&1&-2\\ 0&6&\vrule&1&4}
\end{equation}
\begin{equation}
		  \myvec{1 &0 & \vrule&\frac{1}{6}&\frac{-2}{6}\\ 0&1&\vrule&\frac{1}{6}&\frac{4}{6}}
\end{equation}		       
   
 by calculation we get   
 \begin{equation}
  A^{-1}=\myvec{
      \frac{1}{6} & \frac{-1}{3}\\
    \frac{1}{6} & \frac{2}{3}
     }
  \end{equation}
  \begin{equation}
 6I=\myvec{
    6&0 \\
    0 & \ 6
     }
       \end{equation}
       
 by calculation we get 
 \begin{equation}
 M=\myvec{
    1&-2 \\
    1 & \ 4
     }
     \end{equation}
     
     

\end{document}
