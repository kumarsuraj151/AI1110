\documentclass[journal,12pt,twocolumn]{IEEEtran}

\usepackage{tfrupee}
\usepackage{enumitem}
\usepackage{amsmath}
\usepackage{amssymb}
\newcommand{\myvec}[1]{\ensuremath{\begin{pmatrix}#1\end{pmatrix}}}




\title{Assignment 1}
\author{Suraj kumar \\ \normalsize AI21BTECH11029\\\vspace*{20pt} \Large ICSE 2019 Grade 10}

\setlength{\columnsep}{1cm}
\begin{document}
\maketitle

\textbf{Question 7(c)}\begin{align*} &\myvec{4 & 2 \\-1 & 1 }M=6I\end{align*}
 ,where M is a matrix and I is unit matrix of order $2\times 2.$
		\begin{enumerate}[label=(\roman*)]
		\item State the order of matrix M
		\item Find the matrix M
	\end{enumerate}	
	
 \textbf{Solution (i) } 
 \begin{align}
	 &\Rightarrow   \myvec{
     4 & 2 \\
     -1 & 1 
      }M=6I
 \end{align}
    here I is unit matrix.\\
    we have to find order of matrix M\\
    let the order the matrix M is 
    \begin{align}
    a\times b
\end{align}     
      we know that for multiply two matrix their order must be in the form of  (x,y) (y,z)  here $x,y,z \in N$
      hence order of matrix will be $2\times b$.\\
      so overall left hand side order is
      \begin{align}
      (2\times 2)\times(2\times b) =(2\times b)
\end{align}        
      for comparing LHS=RHS their order must be same so \\
      order of LHS$=(2\times b)$,
      RHS=$(2\times 2)$  hence b = 2.\\
      hence the order of matrix M is $(2\times 2).$\\\\
\textbf{solution (ii)}\\
  \begin{align}
 &A=\myvec{
      4 & 2\\
      -1 & 1
     }
\end{align}
 \begin{align}
 AM=6I
 \end{align}
 
 \centering multiply by $ A^{-1}$
 \begin{align}  
     M=A^{-1}\times 6I\\  \therefore I\times M=M
    \end{align}
    as we know that  $A \times I=I\times A^{-1}$
   	\begin{align}
   	&\myvec{4 &2 & \vrule &1&0\\ -1&1&\vrule&0&1}
	\end{align}\\* 
$R_2\rightarrow 4R_2+R_1$,
\begin{align}
	&\myvec{4 &2 & \vrule &1&0\\ 0&6&\vrule&1&4}
\end{align}
$R_1\rightarrow 3R_1-R_2$
\begin{align}
	 &\myvec{12&0& \vrule &2&-4\\0&6&\vrule&1&4}
\end{align}
$R_1\rightarrow \frac{R_1}{2}$
\begin{align}
&\myvec{6 &0 & \vrule &1&-2\\ 0&6&\vrule&1&4}\\
&\myvec{1 &0 & \vrule&\frac{1}{6}&\frac{-2}{6}\\ 0&1&\vrule&\frac{1}{6}&\frac{4}{6}}\\
  A^{-1}&=\myvec{
      \frac{1}{6} & \frac{-1}{3}\\
    \frac{1}{6} & \frac{2}{3}
     }
\end{align}
by calculation we get   
 \begin{align}
 A^{-1}&=\myvec{
      \frac{1}{6} & \frac{-1}{3}\\
    \frac{1}{6} & \frac{2}{3}
     }\\
     6I&=\myvec{
    6&0 \\
    0 & \ 6
     }
 \end{align}    
by calculation we get 
 \begin{align}
  &M=\myvec{
    1&-2 \\
    1 & \ 4
     }
 \end{align}   
\end{document}
